%----------------------------------------------------------------------------------------
%	PACKAGES AND CONFIGURATION
%----------------------------------------------------------------------------------------

\usepackage[a4paper, hmargin=25mm, vmargin=30mm, top=20mm]{geometry} % Use A4 paper and set margins

\usepackage{mathpazo} % Palatino font

\usepackage{titling}

\usepackage{fancyhdr} % Customize the header and footer

\usepackage{xifthen} % \isempty test

\usepackage{lastpage} % Required for calculating the number of pages in the document

\usepackage[hidelinks]{hyperref} % Colors for links, text and headings

\usepackage{xcolor} % Coloring of text

\usepackage{amssymb} % More symbols
\usepackage{eurosym} % euro symbol

\setcounter{secnumdepth}{0} % Suppress section numbering

\usepackage[ngerman,english]{babel}
\usepackage[utf8]{inputenc} % Required for inputting international characters
\usepackage[T1]{fontenc} % Output font encoding for international characters

% Bibliography
\usepackage[backend=bibtex, style=nature, maxcitenames=999, maxbibnames=999]{biblatex}
\AtEveryCitekey{\clearfield{year}\clearfield{url}} % hide year and url


\usepackage{titlesec} % Customization of titles and their spacing
\titleformat{\section}{\normalfont\large}{\thesection}{0.1em}{}[{\titlerule[0.8pt]}]
\titlespacing{\section}{0pt}{30pt}{5pt}

\fancypagestyle{plain}{\fancyhf{} % custom page style
	\lfoot{\color{darkgray}\theauthor} % left foot: author
	\rfoot{\color{darkgray}\thepage\ / \pageref{LastPage}}} % right foot: page number
\pagestyle{plain} % Use the custom page style through the document
\renewcommand{\headrulewidth}{0pt} % Disable the default header rule
\renewcommand{\footrulewidth}{0pt} % Disable the default footer rule

\setlength\parindent{0pt} % Stop paragraph indentation

% Non-indenting itemize
\newenvironment{itemize-noindent}
{\setlength{\leftmargini}{0em}\begin{itemize}}
{\end{itemize}}

\let\tmpunderline\underline
\renewcommand{\underline}[1]{\tmpunderline{\smash{#1}}}

% allow discarding item separator
\usepackage{enumitem}

% Text width for tabbing environments
\newlength{\smallertextwidth}
\setlength{\smallertextwidth}{\textwidth}
\addtolength{\smallertextwidth}{-2cm}

% Setup table
\usepackage{array}
\newcolumntype{L}[1]{>{\raggedright\let\newline\\\arraybackslash\hspace{0pt}}m{#1}}

\usepackage[outline]{contour}% http://ctan.org/pkg/contour

\usepackage{xstring} % string length


\input{trim_spaces.tex}

%----------------------------------------------------------------------------------------
%	MAIN HEADER COMMAND
%----------------------------------------------------------------------------------------

\renewcommand{\title}[1]{
{\huge{\color{black}#1}}\\ % Header section name and color
\rule{\textwidth}{0.5mm}\\ % Rule under the header
}


%----------------------------------------------------------------------------------------
%	BLOCK COMMAND
%----------------------------------------------------------------------------------------

\newcommand{\blocktitle}[1]{\textbf{\trim@spaces{#1}}}

\newcommand{\block}[6]{%
% left column: time frame
\begin{tabular}{p{0.15\linewidth} l}%
\ifthenelse{\isempty{#2}}%
{#1}%
{%
\begin{tabular}[t]{@{}l@{}}%
#1%
\StrLen{#1}[\fromstringlen]%
\ifthenelse{\fromstringlen > 5}{\\}{}%
 - %
#2%
\end{tabular}%
}%
%
&%
%
% right column
\begin{tabular}[t]{@{}l@{}}%
% header: project, organisation
\makeatletter%
\ifthenelse{\NOT \isempty{#3} \OR \NOT \isempty{#4}}{%
\parbox[t]{0.79\linewidth}{\raggedright %
%	\blocktitle{\trim@spaces{#3}}%
%	\ifthenelse{\NOT \isempty{#3} \AND \NOT \isempty{#4}}%
%	{, }{}%
%	\organization{#4}%
	\organization{#4}%
	\ifthenelse{\NOT \isempty{#3} \AND \NOT \isempty{#4}}%
	{, }{}%
	\position{#3}%
\par}%
}{}%
% line break only if headers and description present
\ifthenelse{\NOT \isempty{#3} \OR \NOT \isempty{#4} \AND \NOT \isempty{#5}}
{\\}{}%
% description
\ifthenelse{\NOT \isempty{#5}}%
{\parbox[t]{0.79\linewidth}{\raggedright \trim@spaces{#5}}}{}%
% technologies
%\ifthenelse{\NOT \isempty{#6}}{%
%\\
%\underline{Technologies}: #6%
%}{}%
\makeatother
\end{tabular}
\end{tabular}
}


%----------------------------------------------------------------------------------------
%	BLOCK COMMAND UTILIES
%----------------------------------------------------------------------------------------

\newcommand{\organization}[1]{%
\blocktitle{\trim@spaces{#1}}}%

\newcommand{\position}[1]{%
\textit{\trim@spaces{#1}}}%

\newcommand{\subprogram}[1]{%
\contourlength{0.01pt}\contournumber{10}\contour{black}{\trim@spaces{#1}}}%


%----------------------------------------------------------------------------------------
%	EDUCATION COMMAND
%----------------------------------------------------------------------------------------

\newcommand{\educationblock}[6]{%
\block%
{#1}%
{#2}%
{#3}%
{#4}%
{#5}%
{#6}%
}


%----------------------------------------------------------------------------------------
%	PUBLICATION COMMAND
%----------------------------------------------------------------------------------------

\newcommand{\publication}[2]{%
\block%
{% 1: Time frame start
#1%
}{% 2: Time frame end
}{% 3: Activity
}{% 4: Organisation
}{% 5: Description
\fullcite{#2}%
\vspace{5pt}%
}{% 6: Technologies
}
}


%----------------------------------------------------------------------------------------
%	AWARD COMMAND
%----------------------------------------------------------------------------------------

\newcommand{\award}[4]%
% 1: year
% 2: Title
% 3: Organisation
% 4: Description
{%
\block%
{#1}
{}
{#2 \ifthenelse{\NOT \isempty{#4}}{\normalfont[\trim@spaces{#4}]}{}}%
{#3}
{}
{}
}


%----------------------------------------------------------------------------------------
%	LANGUAGE COMMAND
%----------------------------------------------------------------------------------------

\newcommand{\languageproficiency}[2]
% 1: Language
% 2: Proficiency
{%
\textbf{#1}
\\
#2
\\
}


%----------------------------------------------------------------------------------------
%	INTEREST COMMAND
%----------------------------------------------------------------------------------------

\newcommand{\interest}[2]
% 1: Title
% 2: Description
{%
\textbf{#1}
\\
#2
\\
}
