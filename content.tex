%----------------------------------------------------------------------------------------
%      HEADER
%----------------------------------------------------------------------------------------
\author{Martin Schrimpf}
\title{\theauthor\space}


%----------------------------------------------------------------------------------------
%	PERSONAL
%----------------------------------------------------------------------------------------
\setAddress{%
	43 Vassar St\\
	Cambridge\\
	MA 02139}
\setEmail{%
%	martin.schrimpf@outlook.com
	msch@mit.edu}
\setMobile{%
	+1 617-586-6748}
\setWeb{%
	http://mschrimpf.com}


%----------------------------------------------------------------------------------------
%	EDUCATION
%----------------------------------------------------------------------------------------
\newcommand{\education}{%
\educationblock%
{% 1: Time frame start
Since 2017%Since 09/2017
}{% 2: Time frame end
}{% 3: Program
PhD program
}{% 4: University
Massachusetts Institute of Technology (MIT)
}{% 5: Description
Department of Brain and Cognitive Sciences.
\\
Advisor: James DiCarlo.
}{% 6: Technologies
}{% 7: Grade
}

\educationblock%
{% 1: Time frame start
2014 %10/2014
}{% 2: Time frame end
2017%05/2017
}{% 3: Program
Master of Science
}{% 4: University
TU \& LMU Munich \& University of Augsburg
}{% 5: Description
Elite Program Software Engineering.
%Structured approaches to Computer Science and creating software, extra courses in Machine Learning.
%Formal Methods, Distributed Systems, Project Management, Databases, Human Computer Interaction. 
%\\
GPA $4.0$ w/ honors.
\\
Thesis completed at \organization{Harvard University}.
}{% 6: Technologies
Java, C++, SQL, QVTO, Maude, CTL, LTL
}

\educationblock%
{% 1: Time frame start
2011 %10/2011
}{% 2: Time frame end
2014%07/2014
}{% 3: Program
Bachelor of Science
}{% 4: University
TU Munich
}{% 5: Description
%Combination of economic fundamentals and computer science with a focus on Information Systems.
%\\
%Study abroad at \organization{Auckland University of Technology} with a project at \organization{CAIR} on landmark-based perceptual mapping, inspired by the hippocampus.
%\\
Program Information Systems.
\\
Thesis completed at the \organization{University of Sydney}.
}{% 6: Technologies
Java, UML, SQL, C++, C, Assembly, ARIA
}

%\educationblock%
%{% 1: Time frame start
%2003
%}{% 2: Time frame end
%2011
%}{% 3: Program
%Abitur
%}{% 4: University
%Gymnasium Dorfen
%}{% 5: Description
%Focus on Mathematics, Computer Science, Economics, English.
%\subprogram{USA-exchange} with the \organization{C.D. Hylton High school} in Virginia
%}{% 7: Grade
%$3.7$
%}
}

\newcommand{\varEducationInfo}{All grades converted into the American 4-point system}


%----------------------------------------------------------------------------------------
%	EXPERIENCE
%----------------------------------------------------------------------------------------
\newcommand{\professional}{%
\block%
{% 1: Time frame start
%05/2017
2017
}{% 2: Time frame end
%08/2017
}{% 3: Position
Deep Learning Intern
}{% 4: Employer
Salesforce Einstein AI
}{% 5: Description
Advisor: Richard Socher.\\
Flexible architecture search for natural language processing with reinforcement learning and predictive function.
}{% 6: Technologies
PyTorch, AWS
}

\block%
{% 1: Time frame start
2016%04/2016
}{% 2: Time frame end
%11/2016
}{% 3: Position
Research Assistant
}{% 4: Employer
Harvard Medical School
}{% 5: Description
Advisor: Gabriel Kreiman.\\
Recurrent computations for the recognition of occluded objects in humans and models;
Robustness of neural networks to weight perturbations;
Role of context for object recognition.
}{% 6: Technologies
Matlab, Python, keras, Theano, Linux, LSF, ECoG
}

\block%
{% 1: Time frame start
2015 %12/2015
}{% 2: Time frame end
2016%04/2016
}{% 3: Position
Research Assistant
}{% 4: Employer
Oracle Labs
}{% 5: Description
Development of an on-demand cluster database module.
%Enabled research teams to flexibly utilize the Oracle RDBMS on the internal cluster by developing an on-demand database module.
}{% 6: Technologies
Linux, LSF, Virtual Machines
}

\block%
{% 1: Time frame start
Since 2015 %2015 %08/2015 // 03/2017
}{% 2: Time frame end
}{% 3: Activity
Co-Founder / Technical Advisor (since 2017)
}{% 4: Organisation
Integreat Digital Factory
}{% 5: Description
Digitization projects in social sector.
Platform for local information distribution to refugees in dozen of German cities.
}{% 6: Technologies
Xamarin (C\#), Android (Java), WordPress (PHP)
}

\block%
{% 1: Time frame start
2012 %07/2012
}{% 2: Time frame end
2015%12/2015
}{% 3: Position
Freelancer
}{% 4: Employer
Martin Schrimpf Software Solutions
}{% 5: Description
Led the development of a document %and workflow 
management system with optical character recognition to make the client company paper-free.
%Software Development and Services - projects include:
%\begin{description}[noitemsep,topsep=0pt,font=\normalfont\itshape\space]
%\item[Greimel IT-Systemhaus GmbH] Led the development of a document and workflow management system with optical character recognition, making the client company effectively paper-free
%%\item[R-Backup Datensicherung GmbH] Developed a multilingual website's front- and backend to administrate partners and customers and to issue invoices
%\item[Promonde JLT] Implemented an advertisement website for Arabic countries with over 10,000 users per day % https://www.youtube.com/watch?v=qP6hNgYzZOg
%\end{description}
}{% 6: Technologies
Java, JavaScript, PHP
}

\block%
{% 1: Time frame start
2015%07/2015
}{% 2: Time frame end
%10/2015
}{% 3: Position
Software Engineering Intern
}{% 4: Employer
Siemens AG
}{% 5: Description
Behavior-driven testing framework to run a test specification written in natural language.% and is now used in three major business areas.
}{% 6: Technologies
Python
}

%\block%
%{% 1: Time frame start
%07/2014
%}{% 2: Time frame end
%11/2014
%}{% 3: Position
%Study abroad
%}{% 4: Employer
%Auckland University of Technology
%}{% 5: Description
%Worked on a landmark-based approach to perceptual mapping, inspired by the hippocampus, at the \organization{Centre for Artificial Intelligence Research}.
%Courses in Artificial Intelligence and Management.
%}{% 6: Technologies
%Java, JavaScript, PHP
%}
}


%----------------------------------------------------------------------------------------
%	PUBLICATIONS
%----------------------------------------------------------------------------------------
\newcommand{\publications}{%

\publication{2018}{bashivan2018continual}

\publication{2018}{arend2018single}

\publication{2018}{kubilius2018}

\publication{2018}{schrimpf2018b}

\publication{2018}{schrimpf2018}

\publication{2018}{TangSchrimpfLotter2018}

\publication{2017}{cheney2017robustness}

\publication{2016}{schrimpf2016}

}


%----------------------------------------------------------------------------------------
%	PRESENTATIONS
%----------------------------------------------------------------------------------------
\newcommand{\presentations}{%

\block%
{% 1: Time frame start
03/2019  % Mar 07, 2019
}{% 2: Time frame end
}{% 3: Activity
}{% 4: Organisation
Center for Brain-Inspired Computing (C-BRIC)
%author={Schrimpf*, Martin and Kubilius*, Jonas and DiCarlo, James J.},
}{% 5: Description
Brain-Score: Which Artificial Neural Network for Object Recognition is most Brain-Like?
}{% 6: Technologies
}

\block%
{% 1: Time frame start
02/2019  % Feb 28-Mar 03
}{% 2: Time frame end
}{% 3: Activity
}{% 4: Organisation
Computational and Systems Neuroscience (Cosyne)
%author={Schrimpf*, Martin and Kubilius*, Jonas and DiCarlo, James J.},
}{% 5: Description
Using Brain-Score to Evaluate and Build Neural Networks for Brain-Like Object Recognition
}{% 6: Technologies
}

\block%
{% 1: Time frame start
02/2019  % Feb 12
}{% 2: Time frame end
}{% 3: Activity
}{% 4: Organisation
Center for Brains, Minds and Machines (CBMM)
}{% 5: Description
Transforming machine learning models into brain models
}{% 6: Technologies
}

\block%
{% 1: Time frame start
09/2018
}{% 2: Time frame end
}{% 3: Activity
}{% 4: Organisation
Cognitive Computational Neuroscience (CCN)
}{% 5: Description
Brain-Score: Which Artificial Neural Network Best Emulates the Brain’s Neural Network?
}{% 6: Technologies
}

\block%
{% 1: Time frame start
02/2018
}{% 2: Time frame end
}{% 3: Activity
}{% 4: Organisation
Tenenbaum Lab, MIT
}{% 5: Description
A Flexible Approach to Automated RNN Architecture Generation
}{% 6: Technologies
}

\block%
{% 1: Time frame start
12/2016
}{% 2: Time frame end
}{% 3: Activity
}{% 4: Organisation
Brains \& Bits, NIPS Workshops
}{% 5: Description
Recurrent computations for pattern completion
}{% 6: Technologies
}

\block%
{% 1: Time frame start
10/2016
}{% 2: Time frame end
}{% 3: Activity
}{% 4: Organisation
Systems Club, Harvard Medical School
}{% 5: Description
Recurrent computations for pattern completion
}{% 6: Technologies
}
}


%----------------------------------------------------------------------------------------
%	GRANTS
%----------------------------------------------------------------------------------------
\newcommand{\grants}{%

\block%
{% 1: Time frame start
2019  % June 04
}{% 2: Time frame end
}{% 3: Activity
Baylor + MIT
}{% 4: Organisation
UG1 (Clinical Research)
}{% 5: Description
% Human electrical stimulation of V1 blind patients
}{% 6: Technologies
}

\block%
{% 1: Time frame start
2019  % June 06
}{% 2: Time frame end
}{% 3: Activity
MIT + IBM
}{% 4: Organisation
IBM Large Project Proposal
}{% 5: Description
% ThreeDWorld
}{% 6: Technologies
}

}


%----------------------------------------------------------------------------------------
%	TEACHING
%----------------------------------------------------------------------------------------
\newcommand{\teaching}{%

\block%
{% 1: Time frame start
2019  % 21 Aug 2019
}{% 2: Time frame end
}{% 3: Activity
}{% 4: Organisation
Harvard-MIT Computational Neuroscience Journal Club
% Theoretical and Computational Neuroscience Journal Club
}{% 5: Description
Deep Networks and PyTorch
}{% 6: Technologies
}

\block%
{% 1: Time frame start
2019  % Feb - May
}{% 2: Time frame end
}{% 3: Activity
}{% 4: Organisation
Neural Mechanisms of Cognitive Computation
}{% 5: Description
Graduate course, Teaching Assistant
}{% 6: Technologies
}

\block%
{% 1: Time frame start
2017  % Oct
}{% 2: Time frame end
}{% 3: Activity
}{% 4: Organisation
MIT BCS Peer Lectures
}{% 5: Description
Introduction to Deep Learning
}{% 6: Technologies
}

}


%----------------------------------------------------------------------------------------
%	AWARDS
%----------------------------------------------------------------------------------------
\newcommand{\awards}{%

\award%
% https://impactchallenge.withgoogle.com/deutschland2018/charities/digital-factory
{% 1: year
2019
}{% 2: Title
Travel award
}{% 3: Organisation
Patrick J. McGovern
}{% 4: Description
%USD 1,000
}

\award%
% https://impactchallenge.withgoogle.com/deutschland2018/charities/digital-factory
{% 1: year
2018
}{% 2: Title
Impact Challenge (Integreat)
}{% 3: Organisation
Google.org
}{% 4: Description
finalist, 250,000\euro{}
}

\award%
{% 1: year
2017
}{% 2: Title
Henry E. Singleton Fellowship
}{% 3: Organisation
MIT
}{% 4: Description
tuition and stipend  % 67705 + 35700
}

\award%
% https://eu-youthaward.org/winning-project/integreat/
% https://www.facebook.com/EuropeanYouthAward/photos/a.179440182233854.1073741828.173261352851737/834546526723213/?type=3&theater
{% 1: year
2017
}{% 2: Title
European Youth Award (Integreat)
}{% 3: Organisation
Council of Europe
}{% 4: Description
winner
}

%\award%
%{% 1: year
%2017
%}{% 2: Title
%Social Impact Award (Integreat)
%}{% 3: Organisation
%TUM School of Management
%}{% 4: Description
%competition winner
%}

\award%
{% 1: year
2016
}{% 2: Title
FITweltweit
}{% 3: Organisation
DAAD German Academic Exchange Service
}{% 4: Description
scholarship for research abroad % 900 EUR/month
}

\award%
{% 1: year
2016
}{% 2: Title
Teilstipendium
}{% 3: Organisation
University of Augsburg
}{% 4: Description
scholarship
}

\award%
{% 1: year
2016
}{% 2: Title
Integrationspreis (Integreat)
}{% 3: Organisation
Government of Swabia
}{% 4: Description
competition winner
}

%\award%
%{% 1: year
%2016
%}{% 2: Title
%Winner Social Society (Integreat)
%}{% 3: Organisation
%Idea- and Startup-competition Generation-D
%}{% 4: Description
%competition winner
%}

\award%
{% 1: year
2015
}{% 2: Title
Deutschlandstipendium
}{% 3: Organisation
Federal Ministry for Education and Research, Roland und Ute Lacher Fonds
}{% 4: Description
scholarship % 300 EUR/month
}

\award%
{% 1: year
2014
}{% 2: Title
Ministeriumsstipendium
}{% 3: Organisation
Bavarian State Ministry %for Education, Science and the Arts
}{% 4: Description
scholarship
}

%\award%
%{% 1: year
%201\{3,4,5,6\}
}{% 2: Title
%e-fellows.net scholarship
%}{% 3: Organisation
}
}


%----------------------------------------------------------------------------------------
%	EXTRACURRICULAR ACTIVITIES
%----------------------------------------------------------------------------------------
\newcommand{\extracurricular}{%
\block%
{% 1: Time frame start
2018%02/2016
}{% 2: Time frame end
}{% 3: Activity
Trainee Leadership Council
}{% 4: Organisation
CBMM (MIT \& Harvard)
}{% 5: Description
}{% 6: Technologies
}

%\block%
%{% 1: Time frame start
%Since 2018
%}{% 3: Activity
%Technical Advisor
%}{% 4: Organisation
%DeepHealth
%}

\block%
{% 1: Time frame start
2016%02/2016
}{% 2: Time frame end
}{% 3: Activity
Organization of AI Workshop
}{% 4: Organisation
University of Augsburg
}{% 5: Description
%Organized two-day workshop. % on Neural Networks, Machine Learning and Organic Computing. 
%Speakers: Prof. Günther Palm, PD Rolf Würtz, and Dr. Joschka Bödecker.
}{% 6: Technologies
}

\block
{% 1: Time frame start
2015
}{% 2: Time frame end
2016
}{% 3: Activity
MINGA Mentor for International Students
}{% 4: Organisation
TU Munich
}{% 5: Description
}{% 6: Technologies
}

%\block%
%{% 1: Time frame start
%2013
%}{% 2: Time frame end
%2016
%}{% 3: Activity
%Rotaract Club München Residenz
%}{% 5: Description
%Youth club of Rotary: community, helping and learning. Social initiatives, e.g. with our orphanage sponsorship
%}
}


%----------------------------------------------------------------------------------------
%      LANGUAGES
%----------------------------------------------------------------------------------------
\newcommand{\languages}{
\languageproficiency{German}{Native proficiency}
\languageproficiency{English}{Full professional proficiency}
\languageproficiency{Japanese}{Elementary proficiency}
\languageproficiency{French}{Elementary proficiency}
}


%----------------------------------------------------------------------------------------
%      INTERESTS
%----------------------------------------------------------------------------------------
\newcommand{\interests}{
\interest{Travelling}{Insights into various cultures in places such as Africa, Australia and India}
\interest{Martial Arts}{Sporty balance, perfection of techniques and meditation with Judo and Shaolin}
\interest{Brain-inspired Computing}{Getting behind the concepts of cognition and intelligence on the basis of biological findings, side projects in e.g. deep reinforcement learning and home automation}
}


%----------------------------------------------------------------------------------------
%	ADVISED STUDENTS
%----------------------------------------------------------------------------------------
\newcommand{\advisedstudents}{%

\block%
{% 1: Time frame start
2019
}{% 2: Time frame end
}{% 3: who
Fukushi Sato
}{% 4: Organisation
TUM / MIT
}{% 5: Description
Building temporal models of the ventral stream
}{% 6: Technologies
}

\block%
{% 1: Time frame start
2018
}{% 2: Time frame end
2019
}{% 3: who
William Hartman
}{% 4: Organisation
MIT
}{% 5: Description
Identifying high-performance substructures within architectures
}{% 6: Technologies
}

\block%
{% 1: Time frame start
Fall 2016
}{% 2: Time frame end
}{% 3: who
Jacklyn Sarette
}{% 4: Organisation
Emmanuel College
}{% 5: Description
Behavioral experiments on visual context
}{% 6: Technologies
}

\block%
{% 1: Time frame start
Fall 2016
}{% 2: Time frame end
}{% 3: who
Doré de Morsier
}{% 4: Organisation
ETH Zurich
}{% 5: Description
Behavioral experiments on the recognition of novel objects
}{% 6: Technologies
}

\block%
{% 1: Time frame start
Summer 2016
}{% 2: Time frame end
}{% 3: who
Wendy Fernandez
}{% 4: Organisation
City University of New York
}{% 5: Description
Behavioral experiments and data analysis on the identification of occluded objects (MIT Summer Research Program)
}{% 6: Technologies
}
}


%----------------------------------------------------------------------------------------
%	REFERENCES
%----------------------------------------------------------------------------------------

\newcommand{\references}{%
\block%
{% 1: Time frame start
}{% 2: Time frame end
}{% 3: who
Prof. Gabriel Kreiman, PhD
}{% 4: Organisation
Children's Hospital Boston, Harvard Medical School
}{% 5: Description
}{% 6: Technologies
}

\iffalse
\block%
{% 1: Time frame start
}{% 2: Time frame end
}{% 3: who
Prof. Uwe Röhm, PhD
}{% 4: Organisation
School of IT, University of Sydney
}{% 5: Description
}{% 6: Technologies
}
\fi

\block%
{% 1: Time frame start
}{% 2: Time frame end
}{% 3: who
Prof. Dr. Helmut Krcmar
}{% 4: Organisation
Computer science in economics, Technical University Munich
}{% 5: Description
}{% 6: Technologies
}

\block%
{% 1: Time frame start
}{% 2: Time frame end
}{% 3: who
Prof. Dr. Alexander Knapp
}{% 4: Organisation
Software and Systems Engineering, Augsburg University
}{% 5: Description
}{% 6: Technologies
}
}
